\chapter{Beispielkapitel}
\label{chp:Beispielkapitel}

Beispiel für \code{einen Codeausschnitt} in einem Text.

Beispiel für \name{einen Eigennamen} in einem Text.

Beispiel für \emphasize{eine Hervorhebung} in einem Text.

\section{Sektion Eins}
\label{sec:Sektion_Eins}

\blindtext

\subsection{Untersektion Eins}
\label{sub:Untersektion_Eins}

Beispiel für eine Abkürzung: \ac{hi}

\begin{lstlisting}[
	language={Python},
	caption={Code},
	label={lst:code},
	float={h!}
]
lst = [1, 2, 3]

for i in lst:
  print(i)
\end{lstlisting}

\begin{figure}[ht!]
	\centering
	\includegraphics[width=0.25\linewidth]{images/example-image.png}
	\caption{Grafik}
	\label{fig:grafik}
\end{figure}

\begin{table}[ht!]
	\selectfont
	\centering
	\begin{tabular}{l|ccc}
		Test & Feld & Feld & Feld \\
		\hline
		Eins & 1 & 2 & 3 \\
		Zwei & 4 & 5 & 6 \\
		Drei & 7 & 8 & 9 \\
	\end{tabular}
	\caption{Tabelle}
	\label{tab:tabelle}
\end{table}

\begin{figure}[ht!]
\centering
\includegraphics[width=0.25\linewidth]{images/example-image.png}
\caption{Grafik}
\label{fig:grafik2}
\end{figure}

\begin{table}[ht!]
	\selectfont
	\centering
	\rowcolors{2}{gray!25}{white}
	\begin{tabular}{ccc}
		\toprule
		\belowrulesepcolor{gray!50}
	    \rowcolor{gray!50}
		& \bfseries Head & \bfseries Head \\
		\aboverulesepcolor{gray!50}
		\midrule
		x & Wert & Wert \\
		x & Wert & Wert \\
		x & Wert & Wert \\
		x & Wert & Wert \\
		\aboverulesepcolor{gray!25}
		\bottomrule
	\end{tabular}
	\caption{Tabelle}
	\label{tab:tabelle2}
\end{table}

Beispiel für eine Quellenangabe \cite{article}.

\index{Hilfe}

Beispiel für mehrere Quellenangaben \cite{book, inproceedings, misc}.

\newpage

\begin{itemize}
	\setlength\itemsep{0pt}
	\item Listenelement 1,
	\item Listenelement 2 und
	\item Listenelement 3.
\end{itemize}

\begin{enumerate}
	\setlength\itemsep{0pt}
	\item Nummeriertes Listenelement 1,
	\item nummeriertes Listenelement 2 und
	\item nummeriertes Listenelement 3.
\end{enumerate}