\chapter{Weitere Beispiele}
\label{chp:Weitere_Beispiele}

\section{Textformatierung}
\label{sec:Textformatierung}

Beispiel für \code{einen Codeausschnitt} in einem Text.

Beispiel für \name{einen Eigennamen} in einem Text.

Beispiel für \emph{eine Hervorhebung} in einem Text.

Text kann auch {\color{red}eingefärbt} werden.
Es empfiehlt sich, für alle im Text verwendeten Hervorhebungsarten jedoch einen eigenen Befehl in style.tex zu definieren.
Dies garantiert Einheitlichkeit, insbesondere im Fall einer Änderung dieser Hervorhebung.

Schriftgrößen sollten im Fließtext nie geändert werden. Für Tabellen und Abbildungen stehen folgende Befehle für entsprechende Fontgröße zur Verfügung:

\begin{table}[ht!]
	\selectfont
	\centering
	\begin{tabular}{|l|r|}
		\hline
		tiny & 6pt \\\hline
		scriptsize & 8pt \\\hline
		footnotesize & 10pt \\\hline
		small & 11pt \\\hline
		normalsize & 12pt \\\hline
		large & 14pt \\\hline
		Large & 17pt \\\hline
		LARGE & 20pt \\\hline
		huge & 25pt \\\hline
		Huge & 25pt \\\hline
	\end{tabular}
	\caption{Schriftgrößen}
	\label{tab:fontsize}
\end{table}

\section{Aufzählungen}
\label{sec:Aufzaehlungen}

\begin{itemize}
	\item Listenelement 1,
	\item Listenelement 2 und
	\item Listenelement 3.
\end{itemize}

Für eine engere Darstellung kann der Abstand verkleinert werden.

\begin{enumerate}
	\setlength\itemsep{0pt}
	\item Nummeriertes Listenelement 1,
	\item nummeriertes Listenelement 2 und
	\item nummeriertes Listenelement 3.
\end{enumerate}

\subsection{Veränderte Aufzählungssymbole}
\label{sub:Veraenderte_Aufzaehlungssymbole}

Mit dem Parameter \emph{label} kann die Aufzählung individuell gestaltet werden.
Möglich sind zum Beispiel

\begin{enumerate}[label=\Roman*.)]
	\setlength\itemsep{0pt}
	\item \textbackslash{}alph* für Klein- und \textbackslash{}Alph* für Großbuchstaben,
	\item \textbackslash{}arabic* für Zahlen und
	\item \textbackslash{}roman* für kleine und \textbackslash{}Roman* für große römische Zahlen.
\end{enumerate}

Mit weiteren Parametern wie z.\,B. \emph{align} und \emph{leftmargin} kann die Darstellung weiter angepasst werden.

\section{Referenzen}
\label{sec:Referenzen}

Kapitel, Schaubilder usw. können über ihr optionales Label referenziert werden.

Was alles in der Referenz stehen soll kann über verschiedene Befehle festgelegt werden.

{ % Stylized tabular environment to look like text instead of like a table
	\def\arraystretch{1.2}
	\begin{tabular}{@{}lcl@{}} % @{} at beginning / end removes default left / right indentation
		\normalsize ref & $\rightarrow$ & \normalsize\ref{sec:Referenzen} \\
		\normalsize nameref & $\rightarrow$ & \normalsize\nameref{sec:Referenzen} \\
		\normalsize autoref & $\rightarrow$ & \normalsize\autoref{sec:Referenzen} \\
		\normalsize fullref & $\rightarrow$ & \normalsize\fullref{sec:Referenzen} \\
	\end{tabular}
}

Labels können auch als \textlabel[TL]{Textlabel (TL)}{text:tl} definiert und später erneut referenziert werden: \ref{text:tl}.

\section{Abkürzungen}
\label{sec:Abkuerzungen}

Abkürzungen werden in der acronyms.tex Datei angelegt.

Beim ersten Aufruf wird der volle Text angezeigt: \ac{HFU}.

Bei erneutem Aufruf wird nurnoch die Abkürzung angezeigt: \ac{HFU}.

\section{Quellen}
\label{sec:Quellen}

Quellen können, wie Referenzen, mit verschiedenen Befehlen referenziert werden.

{
	\def\arraystretch{1.2}
	\begin{tabular}{@{}lcl@{}}
		\normalsize cite & $\rightarrow$ & \normalsize\cite{mustermann12} \\
		\normalsize citeA & $\rightarrow$ & \normalsize\citeA{mustermann12} \\
		\normalsize citeauthor & $\rightarrow$ & \normalsize\citeauthor{mustermann12} \\
		\normalsize citeyear & $\rightarrow$ & \normalsize\citeyear{mustermann12} \\
	\end{tabular}
}

Es können auch mehrere Quellen gleichzeitig zitiert werden \cite{doe01,musterfrau20,normal19}

Alternativ können auch Fußnoten zum Quellenvermerk dienen\,\footnote{\citeA{dupont12}}. % der Befehl \, erzeugt ein geschütztes, halbes Leerzeichen

\subsection{Einträge in bibliography.bib}
\label{sub:Eintraege_in_bibliography.bib}

In \autoref{tab:bibentries} werden die am häufigst benötigten Felder für die jeweiligen Eintragsarten aufgelistet.
Um ein gutes Quellverzeichnis zu erzeugen sollten möglichst viele dieser Felder bei jeder Quelle bestückt werden.
Darüber hinaus finden sich auf der Dokumentation des jeweils verwendeten Zitierstils (hier: apacite) weitere Felder wie \emph{originaltitle} o.\,Ä., welche jedoch selten benötigt werden.
Wie diese Felder sich auswirken, kann im Literaturverzeichnis dieser Demo gesehen werden.

\begin{table}[ht!]
	\centering
	\resizebox*{\linewidth}{!}{
		\begin{tabular}{l|p{3.5cm}|c|c|c|c|c|p{5.5cm}}
			\multirow{2}{*}{Feld} & \multirow{2}{*}{Beschreibung} & \multicolumn{5}{c|}{Entry Type} & \multirow{2}{*}{Anmerkungen} \\
			\cline{3-7}
			& & article & book & inproceedings & techreport & misc & \\
			\hline
			author* & Autor(en) & $\times$ & $\times$ & $\times$ & $\times$ & $\times$ & Mehrere Autoren werden mit \emph{and} getrennt. Autoren können als \name{Vorname Nachname} oder \name{Nachname, Vorname} angegeben werden. \\
			year* & Erscheinungsjahr & $\times$ & $\times$ & $\times$ & $\times$ & $\times$ & \\
			day & Erscheinungstag & $\times$ & & $\times$ & $\times$ & $\times$ & \\
			month & Erscheinungsmonat & $\times$ & & $\times$ & $\times$ & $\times$ & \\
			title* & Titel & $\times$ & $\times$ & $\times$ & $\times$ & $\times$ & Groß- / Kleinschreibung wird entfernt. Kann vermieden werden, indem das komplette Feld in zwei paar geschweifte Klammern geschrieben wird. \\
			englishtitle & Englischer Titel & $\times$ & $\times$ & $\times$ & $\times$ & $\times$ & Optional falls Werk einen übersetzten Titel hat. Groß- / Kleinschreibung wird wie bei \emph{title} entfernt. \\
			booktitle* & Buchtitel & & & $\times$ & & & Titel des Buchs, in dem der Artikel erschienen ist. Groß- / Kleinschreibung wird wie bei \emph{title} entfernt. \\
			editor & Herausgeber & & $\times$ & & & & \\
			translator & Übersetzer & $\times$ & $\times$ & & & & \\
			type & Art des Werks & $\times$ & $\times$ & $\times$ & $\times$ & $\times$ & Bspw. Doktorarbeit, Rezension, Broschüre \\
			journal* & Journal, in dem Artikel erschienen ist & $\times$ & & & & & \\
			volume* & Ausgabe & $\times$ & $\times$ & $\times$ & $\times$ & $\times$ & \\
			number* & Nummer der Ausgabe & $\times$ & $\times$ & & $\times$ & $\times$ & \\
			pages & Referenzierte Seiten & $\times$ & & & & & \\
			howpublished & Veröffentlichungsart & $\times$ & $\times$ & $\times$ & $\times$ & $\times$ & Beschreibung, wie etwas veröffentlicht wurde (z.\,B. Name einer Webseite, \name{Unveröffentlichte Arbeit}, \name{Vorgestellt auf Konferenz}). \\
			address* & Adresse des Verlags / der Institution & & $\times$ & $\times$ & $\times$ & $\times$ & Meist nur Stadt oder bei amerikanischen Verlagen Stadt, Staat. \\
			publisher* & Verlag & & $\times$ & $\times$ & & $\times$ & \\
			institution* & Veröffentlichende Institution & & & & $\times$ & & \\
			\makecell[tl]{urldate* / \\ lastchecked} & Letztes Zugriffsdatum auf URL & $\times$ & $\times$ & $\times$ & $\times$ & $\times$ & Nur in Kombination mit \emph{url} nötig. Felder urldate und lastchecked sind austauschbar. \\
			url* & URL & $\times$ & $\times$ & $\times$ & $\times$ & $\times$ & \\
			note & Notizen & $\times$ & $\times$ & $\times$ & $\times$ & $\times$ & \\
			doi & DOI & $\times$ & $\times$ & $\times$ & $\times$ & $\times$ & \\
			\multicolumn{8}{l}{\rule{0pt}{4ex}*\,Markierte Felder sind besonders wichtig. Alle anderen sind in den meisten Fällen nur optional.}
		\end{tabular}
	}
	\caption{Felder für bibliography.bib}
	\label{tab:bibentries}
\end{table}

\section{Theoreme}
\label{sec:Theoreme}

Für Definitionen, Beispiele o.\,Ä. können sogenannte Theoreme verwendet werden.
Diese sehen beispielsweise wie in \autoref{def:example-definition} oder \autoref{ex:example-example} aus.
Weitere solcher Theoreme können selbst in der Datei style.tex in dem entsprechenden Bereich definiert werden.

\begin{definition}
	Dies ist eine Definition.
	\label{def:example-definition}
\end{definition}

\begin{example}
	Dies ist ein Beispiel.
	\label{ex:example-example}
\end{example}