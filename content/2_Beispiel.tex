\chapter{Weitere Beispiele}
\label{chp:Weitere_Beispiele}

\section{Textformatierung}
\label{sec:Textformatierung}

Beispiel für \code{einen Codeausschnitt} in einem Text.

Beispiel für \name{einen Eigennamen} in einem Text.

Beispiel für \emph{eine Hervorhebung} in einem Text.

Text kann auch {\color{red}eingefärbt} werden.
Es empfiehlt sich, für alle im Text verwendeten Hervorhebungsarten jedoch einen eigenen Befehl in style.tex zu definieren.
Dies garantiert Einheitlichkeit, insbesondere im Fall einer Änderung dieser Hervorhebung.

Schriftgrößen sollten im Fließtext nie geändert werden. Für Tabellen und Abbildungen stehen folgende Befehle für entsprechende Fontgröße zur Verfügung:

\begin{table}[ht!]
	\selectfont
	\centering
	\begin{tabular}{|l|r|}
		\hline
		tiny & 6pt \\\hline
		scriptsize & 8pt \\\hline
		footnotesize & 10pt \\\hline
		small & 11pt \\\hline
		normalsize & 12pt \\\hline
		large & 14pt \\\hline
		Large & 17pt \\\hline
		LARGE & 20pt \\\hline
		huge & 25pt \\\hline
		Huge & 25pt \\\hline
	\end{tabular}
	\caption{Schriftgrößen}
	\label{tab:fontsize}
\end{table}

\section{Aufzählungen}
\label{sec:Aufzaehlungen}

\begin{itemize}
	\item Listenelement 1,
	\item Listenelement 2 und
	\item Listenelement 3.
\end{itemize}

Für eine engere Darstellung kann der Abstand verkleinert werden.

\begin{enumerate}
	\setlength\itemsep{0pt}
	\item Nummeriertes Listenelement 1,
	\item nummeriertes Listenelement 2 und
	\item nummeriertes Listenelement 3.
\end{enumerate}

\subsection{Veränderte Aufzählungssymbole}
\label{sub:Veraenderte_Aufzaehlungssymbole}

Mit dem Parameter \emph{label} kann die Aufzählung individuell gestaltet werden.
Möglich sind zum Beispiel

\begin{enumerate}[label=\Roman*.)]
	\setlength\itemsep{0pt}
	\item \textbackslash{}alph* für Klein- und \textbackslash{}Alph* für Großbuchstaben,
	\item \textbackslash{}arabic* für Zahlen und
	\item \textbackslash{}roman* für kleine und \textbackslash{}Roman* für große römische Zahlen.
\end{enumerate}

Mit weiteren Parametern wie z.\,B. \emph{align} und \emph{leftmargin} kann die Darstellung weiter angepasst werden.

\section{Referenzen}
\label{sec:Referenzen}

Kapitel, Schaubilder usw. können über ihr optionales Label referenziert werden.

Was alles in der Referenz stehen soll kann über verschiedene Befehle festgelegt werden.

\begin{tabular}{@{}ll@{}}
  ref & \ref{sec:Referenzen} \\
  nameref & \nameref{sec:Referenzen} \\
  autoref & \autoref{sec:Referenzen} \\
  fullref & \fullref{sec:Referenzen} \\
\end{tabular}

Labels können auch als \textlabel[TL]{Textlabel (TL)}{text:tl} definiert und später erneut referenziert werden: \ref{text:tl}.

\section{Abkürzungen}
\label{sec:Abkuerzungen}

Abkürzungen werden in der acronyms.tex Datei angelegt.

Beim ersten Aufruf wird der volle Text angezeigt: \ac{HFU}.

Bei erneutem Aufruf wird nurnoch die Abkürzung angezeigt: \ac{HFU}.

\section{Quellen}
\label{sec:Quellen}

Quellen können, wie Referenzen, mit verschiedenen Befehlen referenziert werden.

\begin{tabular}{@{}ll@{}}
  cite & \cite{mustermann20} \\
  citeA & \citeA{mustermann20} \\
  citeauthor & \citeauthor{mustermann20} \\
  citeyear & \citeyear{mustermann20} \\
\end{tabular}

Es können auch mehrere Quellen gleichzeitig zitiert werden \cite{musterfrau99,doe12,dupont96}

Alternativ können auch Fußnoten zum Quellenvermerk dienen \footnote{Beispiel für eine Fußnote}.