%----------------------------------------------------------
% Styles
%----------------------------------------------------------

\newcommand{\code}[1]{{\ttfamily #1}}
\newcommand{\name}[1]{{,,#1``}}
\newcommand{\emphasize}[1]{{\itshape #1}}

%----------------------------------------------------------
% Font
%----------------------------------------------------------

% Change font to sans serif
%\renewcommand{\familydefault}{\sfdefault}

\usepackage{gentium}
\usepackage{inconsolata}

%----------------------------------------------------------
% Source code (listings)
%----------------------------------------------------------

% Code style colors
\definecolor{lstbgr}{RGB}{245,245,245}
\definecolor{lstkeyword}{RGB}{0,0,255}
\definecolor{lststring}{RGB}{163,21,21}
\definecolor{lstcomment}{RGB}{0,128,0}

% Default format
\lstset{
	language=Python,%						the language of the code
	backgroundcolor=\color{lstbgr},%		choose the background color
	basicstyle=\ttfamily\footnotesize,%		the size of the fonts that are used for the code
	keywordstyle=\color{lstkeyword},%		keyword style
	stringstyle=\color{lststring},%			the style that is used for strings
	commentstyle=\color{lstcomment},%		comment style
	numberstyle=\ttfamily\footnotesize,%	the style that is used for the line-numbers
	tabsize=2,%								sets default tabsize to 2 spaces
	keepspaces=true,%						keeps spaces in text
	showstringspaces=false,%				use a symbol for spaces in strings
	breakatwhitespace=false,%				sets if automatic breaks should only happen at whitespace
	breaklines=true,%						sets automatic line breaking
	frame=single,%							adds a frame around the code
	captionpos=b,%							where to put the caption
	numbers=left,%							where to put the line-numbers
	numbersep=12pt,%						how far the line-numbers are from the code
	xleftmargin=26.2pt,%					sets left margin for entire box to equal table margin
	framexleftmargin=23pt,%					sets left margin for frame to also surround line numbers
	framexrightmargin=-3.2pt,%				sets left margin for frame to be equal on both sides
}

% Define unformatted language style
\lstdefinelanguage{none}
{
	identifierstyle=
}

%----------------------------------------------------------
% Define new words with hyphenation
%----------------------------------------------------------

\hyphenation{Back-slash}
\hyphenation{Tool-um-ge-bung}
\hyphenation{Tool-um-ge-bun-gen}